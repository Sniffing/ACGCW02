%----------------------------------------------------------------------------------------
%	PACKAGES AND OTHER DOCUMENT CONFIGURATIONS
%----------------------------------------------------------------------------------------

\documentclass{article}

\usepackage{lastpage} % Required to determine the last page for the footer
\usepackage{extramarks} % Required for headers and footers
\usepackage{graphicx} % Required to insert images
\usepackage{lipsum} % Used for inserting dummy 'Lorem ipsum' text into the template
\usepackage{hyperref}

\linespread{1.1} % Line spacing
\usepackage[total={6in, 9in}]{geometry}

\setlength\parindent{0pt} % Removes all indentation from paragraphs



\begin{document}
\title{Advanced Computer Graphics\\ Coursework 2}
\author{Terence Tse, Zhou Yu \\ Team JT}
\maketitle
\newpage

\section{Part 1: Fresnel reflectance}
Using the equations outlined in the notes for Fresnel reflectance,
the graph depicting parallel and perpendicular components of air to glass
(1.0 refraction index to 1.5) can be seen in XXXXXXTODOXXXXXXXX. The
graph of Fresnel Reflectance from glass to air (1.5 to 1.0) is shown in 
XXXXXTODOXXXXXXXXX. The curve for Schlick's approximation of Fresnel
reflectance is named XXXTODOXXX. These can all be found in the "part1" folder
in the pictures folder of the attached files.\\
\\
From air to glass, we took the parallel component measurements and found
Brewster's angle to be $XXXXXXTODOXXXXXXX$.\\
\\
From glass to air, we took the parallel component measurements and from here
could owrk out the critical angle to be $XXXXTODOXXXX$.\\
\\
Finally, with Schlick's approximation, we made sure to use the reflectance
of normal incidence from the air to glass Fresnel reflectance graph which 
was $2.0$.

\section{Part 2: Environment Map(EM) sample generation}
The samples we took from the environment map can be see in XXXTODOXXX.pfm which
is in the "part2" folder of the pictures folder in the attached files. For 
each sample pixel, we have surrounded it with a XXXXXXXXTODOXXXXXXXX 
neighbourhood for better visualisation.\\
\\
We have done sampling with the number of samples, N, being 64, 256 and 1024.
These are "Sampling64.ppm", "Sampling256.ppm" and "Sampling1024.ppm". We have
gamma corrected these too.

\section{Part 3: Environment Map(EM) Sphere Rendering}



\section{Part 4: Rendering a Sphere with Grace EM and PBRT}
Albedo was set at 1.0 and the three sphere can be seen in the
part4 folder of the pictures folder attached. These are the
.pfm and .ppm images named "PBRTSphere8", 
"PBRTSphere16" and "PBRTSphere32" for the 8,16, and
32 sample msampling methods.\\
\\
As we can see from these images XXXXXXXXXXXXXXXXXXXXXXXXTODOXXXXXXXXXXX
%Something about the noise decreasing or some shit

\end{document}
